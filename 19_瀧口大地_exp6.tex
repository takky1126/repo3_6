\documentclass[12pt]{ltjarticle}
\usepackage{enumitem}
\usepackage{caption}
\usepackage{graphicx}
\usepackage[hyphens]{url}
\usepackage{multirow}
\usepackage{float}
\usepackage{amsmath}
\usepackage{here}
\usepackage{array}
\usepackage{geometry}
\usepackage{microtype}
\usepackage{fontspec}
\usepackage{luatexja-fontspec} % 明示的に読み込む
\usepackage{luatexja}
\usepackage{titlesec}
\usepackage{subcaption}
\usepackage{listings}
\usepackage{ascmac}

% ========== listings の設定 ==========
\lstset{
  % フレーム (枠線) の指定:上 (t) と 下 (b) の線だけ
  frame=tb, 
  % 枠線とコードの間の余白
  framesep=5pt, 
  % 等幅フォントで表示
  basicstyle=\ttfamily\small, 
  % 行番号を左側に表示
  numbers=left,
  % 行番号の書式を小さく(tiny)する
  numberstyle=\tiny, 
  % 行番号とコードの間のスペース
  numbersep=5pt, 
  % 行番号の増分 (1行ごと)
  stepnumber=1, 
}

% \usepackage{array}
% \usepackage{booktabs}
% \geometry{margin=20mm}
\titleformat{\section}[block]
  {\normalfont\LARGE\bfseries}
  {第\thesection 章}{1em}{}

\setmainfont{ipaexm.ttf} % 日本語フォント(IPAex明朝)を指定 ※ XeLaTeX or LuaLaTeXが必要
\geometry{left=25mm, right=25mm, top=30mm, bottom=30mm} % 余白設定
\renewcommand{\baselinestretch}{1.2} % 行間を1.2倍に設定

\counterwithin{figure}{section} % 図番号を「1.1」のようにセクションと連動
\counterwithin{table}{section}  % 表番号も同様にセクションと連動

\makeatletter
% section 再定義(改ページなし・見た目調整)
\renewcommand{\section}{%
  \@startsection{section}{1}{\z@}%
  {-2.5ex \@plus -1ex \@minus -.2ex}%
  {1.5ex \@plus.2ex}%
  {\normalfont\LARGE\bfseries\raggedright}%
}

% subsection 再定義(見た目だけ調整,番号はそのまま)
\renewcommand{\subsection}{%
  \@startsection{subsection}{2}{\z@}%
  {-1.5ex \@plus -0.5ex \@minus -.2ex}%
  {1ex \@plus .2ex}%
  {\normalfont\large\bfseries\raggedright}%
}

% section 番号だけ「第n章」に変更
\renewcommand{\thesection}{第\arabic{section}章}

% subsection 番号は従来の n.m 形式を保つ(これが重要)
\renewcommand{\thesubsection}{\arabic{section}.\arabic{subsection}}
\makeatother

\makeatletter
% 表番号を「2.1」「2.2」のようにセクション番号と連動
\renewcommand{\thetable}{\arabic{section}.\arabic{table}}
% キャプションの前に「表」を表示
\renewcommand{\tablename}{表}
% 参照の際にも「表」と表示
\renewcommand{\refname}{\tablename}
\renewcommand{\thefigure}{\arabic{section}.\arabic{figure}}
\renewcommand{\figurename}{図}
\makeatother

% キャプションの書式設定
\captionsetup[figure]{labelfont={bf}, labelsep=space, font=small}  % キャプションを小さくし,「図」を太字に設定

% タイトルと著者の設定
\title{\huge 第5章 サイバーセキュリティ基礎実験2}
\author{\large 3年情報工学科 19番 瀧口大地}
  
\date{} % 日付を空にする

\begin{document}

  \begin{titlepage}
    \vspace*{\fill}
    \begin{center}
      {\Large 情報工学実験 II 第5回レポート}\\
      \vspace{0.5\baselineskip}
      {\Huge サイバーセキュリティ基礎実験2}
    \end{center}

    \vspace{2cm}
    \begin{LARGE}
      \begin{center}
        3年 情報工学科 19番 瀧口大地
      \end{center}
    \end{LARGE}

    \begin{large}
    \vspace{1.5cm}
    \begin{flushleft}
      \normalsize
      提出期限: 2025年12月26日09:00\\
      提出日: 2025年12月22日23:00
    \end{flushleft}

    \vspace{1cm}
    \begin{flushleft}
      共同実験者:3班\\
       4番 井村周慈\\
       9番 カリラ\\
      14番 後藤輝一\\
      30番 三原瑚桜\\
      34番 山口紗音
    \end{flushleft}
    \end{large}

    \vspace*{\fill}
  \end{titlepage}

  \clearpage
  \setcounter{page}{1}
  \pagestyle{plain}

  \newpage
  \noindent
  \begin{huge}
    アブストラクト\\\\
  \end{huge}
   本実験では,Python(パイソン)と呼ばれるスクリプト言語の文法や基本的な構造について
  学習し,Pythonを用いて様々なデータ整理や加工,テキストファイル処理及びツールの基本
  的な作成方法について習得すること,他のプログラミング言語と比較してPythonの特徴や
  利点についても学ぶことを目的とする.具体的にはPythonの基本文法を学ぶため,演算プ
  ログラム,九九表の出力プログラムを作成する.また,Pythonで用いられるモジュールを用
  いて日数計算プログラムを作成する.さらに,テキストファイルを読み込み,ソートするプ
  ログラム,重複データを削除するプログラムを作成する.それらの総括として,アクセスロ
  グから正規表現を用いてSQLコマンドインジェクションを検出するプログラム,各IPアド
  レスのアクセス回数を棒グラフとして出力するプログラムを作成する.また,発展的な課題
  としてmatplotlib を用いて,各IPアドレスのアクセス数を棒グラフとして出力するプログ
  ラムを作成する.具体的な方法として,重複データを削除するプログラムでは重複している
  かどうかの判定をsetを用いて行った.また,アクセスログから正規表現を用いてSQLコマ
  ンドインジェクションを検出するプログラムでは,SQLコマンドインジェクションで見られ
  る特徴的な文字列を正規表現で検出することで,検出を行った.また,各IPアドレスのア
  クセス回数を棒グラフとして出力するプログラムでは,辞書を用いて各IPアドレスのアク
  セス回数をカウントしopenpyxlを用いてエクセルファイルに棒グラフを出力した.同様に
  matplotlib を用いて各 IP アドレスごとのアクセス数を棒グラフとして出力した.作成した
  これらのプログラムは実験時間内に作成し,正常に動作することが確認できた.

  \newpage
  \noindent
  % \setcounter{page}{1}
  \section{実験目的}
   本実験では,Python(パイソン)と呼ばれるスクリプト言語の文法や基本的な構造について
  学習し,Pythonを用いて様々なデータ整理や加工,テキストファイル処理及びツールの基本
  的な作成方法について習得することを目的とする.また,他のプログラミング言語と比較し
  てPython の特徴や利点についても学ぶ.
  
  \newpage
  \section{実験結果}
   指導書に示されたプログラミング課題をそれぞれアイデア,プログラム,結果に分けて
  述べていく.

  \subsection{スクリプトプログラミングの学習}
  \subsubsection{アイデア}
  Pythonを用いて簡単な演算を行う.演算子と役割は以下の表\ref{pythonでの四則演算子}
  に示すとおりである.
  
  \begin{table}[H]
    \centering
    \caption{pythonでの四則演算子}
    \label{pythonでの四則演算子}
    \begin{tabular}{|c|l|}
    \hline
    +  & 加算 \\ \hline
    -  & 減算 \\ \hline
    * & 乗算 \\ \hline
    /  & 除算 \\ \hline
    \% & 剰余 \\ \hline
    // & 除算(切り捨て) \\ \hline
  \end{tabular}
  \end{table}

  これらの演算子を用いて指導書に示されたサンプルプログラムを実行する.

  \subsubsection{プログラム}
  以下に指導書のサンプルプログラムを示す.

  \begin{figure}[H]
  \begin{lstlisting}[title={サンプルプログラム}]
  print("10 + 3 =", 10+3)
  print("10 - 3 =", 10-3)
  print("10 * 3 =", 10*3)
  print("10 / 3 =", 10/3)
  print("10 % 3 =", 10%3)
  print("10 // 3 =", 10//3)
  \end{lstlisting}
  \end{figure}

  サンプルプログラムでは1行目から順に10と3の加算,減算,乗算,除算,剰余,切り捨て除算
  を行っている.

  \subsubsection{結果}
  サンプルプログラムを実行した結果を図\ref{サンプルプログラムの実行結果}に示す.

  \begin{figure}[H]
    \begin{center}
    \begin{screen}
    \begin{verbatim}
    >python four_arithmetic.py
    10 + 3 = 13
    10 - 3 = 7
    10 * 3 = 30
    10 / 3 = 3.3333333333333335
    10 % 3 = 1
    10 // 3 = 3
    \end{verbatim}
    \end{screen}
    \end{center} 
    \caption{サンプルプログラムの実行結果} 
    \label{サンプルプログラムの実行結果}
  \end{figure}

  図\ref{サンプルプログラムの実行結果}より,表\ref{pythonでの四則演算子}
  で示した通りに動作していることがわかる.よって指導書に示されたサンプルプログラムは
  正しく動作したと言える.

  \subsection{九九表の出力}
  ここでは図\ref{出力したい九九の表}のような九九表を出力するプログラムを作成する.

  \begin{figure}[H]
    \begin{center}
    \begin{screen}
    \begin{verbatim}
       |  1  2  3  4  5  6  7  8  9
    ---+---------------------------
      1|  1  2  3  4  5  6  7  8  9
      2|  2  4  6  8 10 12 14 16 18
      3|  3  6  9 12 15 18 21 24 27
      4|  4  8 12 16 20 24 28 32 36
      5|  5 10 15 20 25 30 35 40 45
      6|  6 12 18 24 30 36 42 48 54
      7|  7 14 21 28 35 42 49 56 63
      8|  8 16 24 32 40 48 56 64 72
      9|  9 18 27 36 45 54 63 72 81
    \end{verbatim}
    \end{screen}
    \end{center} 
    \caption{出力したい九九の表} 
    \label{出力したい九九の表}
  \end{figure}

  \subsubsection{アイデア}
  図\ref{出力したい九九の表}の最初の2行はそれぞれ掛ける数と区切りの横線を出力
  しているだけである.3行目以降は掛ける数と同じような形で九九表が出力されているため,
  指導書の指定の通りにフォーマットを使うとコンパクトなコーディングができる.
  また,各数値は1桁のものでも2桁のものでも2桁相当の空間を取って出力するため,これも
  フォーマットを活用できる.

  \subsubsection{プログラム}
  アイデアに則って組んだ九九表を出力するプログラムを以下に示す.

  \begin{figure}[H]
  \begin{lstlisting}[title={九九表を出力するプログラム}]
    # フォーマット
    f = " {0} |"
    n = " {0:2}"

    # 1行目
    print(f.format(" "),end='')
    for i in range(1,10):
        st = " " + (str)(i)
        print(n.format(st),end='')

    # 2行目
    print("\n"+"---+----------------------------")

    # 3行目以降
    for i in range(1,10):
        print(f.format(i),end='')
        for j in range(1,10):
            print(n.format(i*j),end='')
        print()
  \end{lstlisting}
  \end{figure}

  コメントにある通り,2~3行目がフォーマット,6~9行目が出力の1行目,
  12行目が出力の2行目,15~19行目が出力の3行目以降を処理している.

  \subsubsection{結果}
  図\ref{k001.pyの実行結果}に実行結果を示す.図\ref{k001.pyの実行結果}の九九表
  が図\ref{出力したい九九の表}と一致していることから,プログラムは正しく動作したと言える.
  
  \begin{figure}[H]
    \begin{center}
    \begin{screen}
    \begin{verbatim}
    >python k001.py
       |  1  2  3  4  5  6  7  8  9
    ---+---------------------------
      1|  1  2  3  4  5  6  7  8  9
      2|  2  4  6  8 10 12 14 16 18
      3|  3  6  9 12 15 18 21 24 27
      4|  4  8 12 16 20 24 28 32 36
      5|  5 10 15 20 25 30 35 40 45
      6|  6 12 18 24 30 36 42 48 54
      7|  7 14 21 28 35 42 49 56 63
      8|  8 16 24 32 40 48 56 64 72
      9|  9 18 27 36 45 54 63 72 81
    \end{verbatim}
    \end{screen}
    \end{center} 
    \caption{k001.pyの実行結果}
    \label{k001.pyの実行結果}
  \end{figure}

  \subsection{閏年判定}
  ここでは入力された年が閏年かどうかを判定するプログラムを作成する.

  \subsubsection{アイデア[1]}
  Pythonにはcalenderモジュールというものがあり,これを用いることで
  1行のコードで与えられた数値型の年が閏年かどうかを判定できる.このモジュールを
  用いて本課題を解決する.

  \subsubsection{コード}
  前述のcalenderモジュールを用いて作成したコードを以下に示す.

  \begin{figure}[H]
  \begin{lstlisting}[title={閏年判定プログラム}]
    import calendar

    year = input("何年(YYYY):")
    year_i = int(year)

    if calendar.isleap(year_i):
        print(year_i , "年は閏年です。")
    else:
        print(year_i , "年は平年です。")
  \end{lstlisting}
  \end{figure}

  コードの1行目ではcalenderモジュールをインポートしており,3行目で年数の入力処理,
  4行目で入力の数値型変換,6行目以降でcalenderモジュールを用いた閏年判定と出力
  を行っている.

  \subsubsection{結果}
  以下の図\ref{k011.pyの実行結果}にプログラムで示したコードを実行した結果を示す.
  平年と閏年をそれぞれ入力し,正しい結果が出力されるか試した.

  \begin{figure}[H]
    \begin{center}
    \begin{screen}
    \begin{verbatim}
    > python test.py
    何年(YYYY):2025
    2025 年は平年です。
    > python test.py
    何年(YYYY):2020
    2020 年は閏年です。
    \end{verbatim}
    \end{screen}
    \end{center} 
    \caption{k011.pyの実行結果}
    \label{k011.pyの実行結果}
  \end{figure}

  図\ref{k011.pyの実行結果}の出力結果から,閏年を閏年,平年を平年と出力できたことが
  わかる.このことから本実験のプログラムは正しく動作したと言える.

  \subsection{ファイル操作}
  本実験ではtemperature.datファイルを読み込み,昇順か降順かしいていされた順にソート
  するプログラムを作成する.

  \subsubsection{アイデア}


  \newpage
  \begin{huge}
    参考文献\\\\
  \end{huge}
  % \noindent[]ページ名,\url{},2026年01/11参照.\\
  \noindent[1]Pythonでうるう年を判定・カウント・列挙,\url{https://note.nkmk.me/python-calendar-leap-year/},2026年01/11参照.\\
  
  \newpage
  \begin{huge}
    感想\\\\
  \end{huge}
\end{document}
