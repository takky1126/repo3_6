\documentclass[12pt]{ltjarticle}
\usepackage{enumitem}
\usepackage{caption}
\usepackage{graphicx}
\usepackage[hyphens]{url}
\usepackage{multirow}
\usepackage{float}
\usepackage{amsmath}
\usepackage{here}
\usepackage{array}
\usepackage{geometry}
\usepackage{microtype}
\usepackage{fontspec}
\usepackage{luatexja-fontspec} % 明示的に読み込む
\usepackage{luatexja}
\usepackage{titlesec}
\usepackage{subcaption}
\usepackage{listings}
\usepackage{ascmac}

% ========== listings の設定 ==========
\lstset{
  % フレーム (枠線) の指定:上 (t) と 下 (b) の線だけ
  frame=tb, 
  % 枠線とコードの間の余白
  framesep=5pt, 
  % 等幅フォントで表示
  basicstyle=\ttfamily\small, 
  % 行番号を左側に表示
  numbers=left,
  % 行番号の書式を小さく(tiny)する
  numberstyle=\tiny, 
  % 行番号とコードの間のスペース
  numbersep=5pt, 
  % 行番号の増分 (1行ごと)
  stepnumber=1, 
}

% \usepackage{array}
% \usepackage{booktabs}
% \geometry{margin=20mm}
\titleformat{\section}[block]
  {\normalfont\LARGE\bfseries}
  {第\thesection 章}{1em}{}

\setmainfont{ipaexm.ttf} % 日本語フォント(IPAex明朝)を指定 ※ XeLaTeX or LuaLaTeXが必要
\geometry{left=25mm, right=25mm, top=30mm, bottom=30mm} % 余白設定
\renewcommand{\baselinestretch}{1.2} % 行間を1.2倍に設定

\counterwithin{figure}{section} % 図番号を「1.1」のようにセクションと連動
\counterwithin{table}{section}  % 表番号も同様にセクションと連動

\makeatletter
% section 再定義(改ページなし・見た目調整)
\renewcommand{\section}{%
  \@startsection{section}{1}{\z@}%
  {-2.5ex \@plus -1ex \@minus -.2ex}%
  {1.5ex \@plus.2ex}%
  {\normalfont\LARGE\bfseries\raggedright}%
}

% subsection 再定義(見た目だけ調整,番号はそのまま)
\renewcommand{\subsection}{%
  \@startsection{subsection}{2}{\z@}%
  {-1.5ex \@plus -0.5ex \@minus -.2ex}%
  {1ex \@plus .2ex}%
  {\normalfont\large\bfseries\raggedright}%
}

% section 番号だけ「第n章」に変更
\renewcommand{\thesection}{第\arabic{section}章}

% subsection 番号は従来の n.m 形式を保つ(これが重要)
\renewcommand{\thesubsection}{\arabic{section}.\arabic{subsection}}
\makeatother

\makeatletter
% 表番号を「2.1」「2.2」のようにセクション番号と連動
\renewcommand{\thetable}{\arabic{section}.\arabic{table}}
% キャプションの前に「表」を表示
\renewcommand{\tablename}{表}
% 参照の際にも「表」と表示
\renewcommand{\refname}{\tablename}
\renewcommand{\thefigure}{\arabic{section}.\arabic{figure}}
\renewcommand{\figurename}{図}
\makeatother

% キャプションの書式設定
\captionsetup[figure]{labelfont={bf}, labelsep=space, font=small}  % キャプションを小さくし,「図」を太字に設定

% タイトルと著者の設定
\title{\huge 第5章 サイバーセキュリティ基礎実験2}
\author{\large 3年情報工学科 19番 瀧口大地}
  
\date{} % 日付を空にする

\begin{document}

  \begin{titlepage}
    \vspace*{\fill}
    \begin{center}
      {\Large 情報工学実験 II 第5回レポート}\\
      \vspace{0.5\baselineskip}
      {\Huge サイバーセキュリティ基礎実験2}
    \end{center}

    \vspace{2cm}
    \begin{LARGE}
      \begin{center}
        3年 情報工学科 19番 瀧口大地
      \end{center}
    \end{LARGE}

    \begin{large}
    \vspace{1.5cm}
    \begin{flushleft}
      \normalsize
      提出期限: 2025年12月26日09:00\\
      提出日: 2025年12月22日23:00
    \end{flushleft}

    \vspace{1cm}
    \begin{flushleft}
      共同実験者:3班\\
       4番 井村周慈\\
       9番 カリラ\\
      14番 後藤輝一\\
      30番 三原瑚桜\\
      34番 山口紗音
    \end{flushleft}
    \end{large}

    \vspace*{\fill}
  \end{titlepage}

  \clearpage
  \setcounter{page}{1}
  \pagestyle{plain}

  \newpage
  \noindent
  \begin{huge}
    アブストラクト\\\\
  \end{huge}
   本実験では,Python(パイソン)と呼ばれるスクリプト言語の文法や基本的な構造について
  学習し,Pythonを用いて様々なデータ整理や加工,テキストファイル処理及びツールの基本
  的な作成方法について習得すること,他のプログラミング言語と比較してPythonの特徴や
  利点についても学ぶことを目的とする.具体的にはPythonの基本文法を学ぶため,演算プ
  ログラム,九九表の出力プログラムを作成する.また,Pythonで用いられるモジュールを用
  いて日数計算プログラムを作成する.さらに,テキストファイルを読み込み,ソートするプ
  ログラム,重複データを削除するプログラムを作成する.それらの総括として,アクセスロ
  グから正規表現を用いてSQLコマンドインジェクションを検出するプログラム,各IPアド
  レスのアクセス回数を棒グラフとして出力するプログラムを作成する.また,発展的な課題
  としてmatplotlib を用いて,各IPアドレスのアクセス数を棒グラフとして出力するプログ
  ラムを作成する.具体的な方法として,重複データを削除するプログラムでは重複している
  かどうかの判定をsetを用いて行った.また,アクセスログから正規表現を用いてSQLコマ
  ンドインジェクションを検出するプログラムでは,SQLコマンドインジェクションで見られ
  る特徴的な文字列を正規表現で検出することで,検出を行った.また,各IPアドレスのア
  クセス回数を棒グラフとして出力するプログラムでは,辞書を用いて各IPアドレスのアク
  セス回数をカウントしopenpyxlを用いてエクセルファイルに棒グラフを出力した.同様に
  matplotlib を用いて各 IP アドレスごとのアクセス数を棒グラフとして出力した.作成した
  これらのプログラムは実験時間内に作成し,正常に動作することが確認できた.

  \newpage
  \noindent
  % \setcounter{page}{1}
  \section{実験目的}
   本実験では,Python(パイソン)と呼ばれるスクリプト言語の文法や基本的な構造について
  学習し,Pythonを用いて様々なデータ整理や加工,テキストファイル処理及びツールの基本
  的な作成方法について習得することを目的とする.また,他のプログラミング言語と比較し
  てPython の特徴や利点についても学ぶ.
  
  \newpage
  \section{実験結果}
   指導書に示されたプログラミング課題をそれぞれアイデア,プログラム,結果に分けて
  述べていく.

  \subsection{スクリプトプログラミングの学習(6.10.1)}
  \subsubsection{アイデア}
   本実験課題ではPythonを用いて簡単な演算を行う.演算子と役割は以下の表
  \ref{pythonでの四則演算子}に示すとおりである.
  
  \begin{table}[H]
    \centering
    \caption{pythonでの四則演算子}
    \label{pythonでの四則演算子}
    \begin{tabular}{|c|l|}
    \hline
    +  & 加算 \\ \hline
    -  & 減算 \\ \hline
    * & 乗算 \\ \hline
    /  & 除算 \\ \hline
    \% & 剰余 \\ \hline
    // & 除算(切り捨て) \\ \hline
  \end{tabular}
  \end{table}

  これらの演算子を用いて指導書に示されたサンプルプログラムを実行する.

  \subsubsection{プログラム}
  以下に指導書のサンプルプログラムを示す.

  \begin{figure}[H]
  \begin{lstlisting}[title={サンプルプログラム}]
  print("10 + 3 =", 10+3)
  print("10 - 3 =", 10-3)
  print("10 * 3 =", 10*3)
  print("10 / 3 =", 10/3)
  print("10 % 3 =", 10%3)
  print("10 // 3 =", 10//3)
  \end{lstlisting}
  \end{figure}

  サンプルプログラムでは1行目から順に10と3の加算,減算,乗算,除算,剰余,切り捨て除算
  を行っている.

  \subsubsection{結果}
  サンプルプログラムを実行した結果を図\ref{サンプルプログラムの実行結果}に示す.

  \begin{figure}[H]
    \begin{center}
    \begin{screen}
    \begin{verbatim}
    >python four_arithmetic.py
    10 + 3 = 13
    10 - 3 = 7
    10 * 3 = 30
    10 / 3 = 3.3333333333333335
    10 % 3 = 1
    10 // 3 = 3
    \end{verbatim}
    \end{screen}
    \end{center} 
    \caption{サンプルプログラムの実行結果} 
    \label{サンプルプログラムの実行結果}
  \end{figure}

  図\ref{サンプルプログラムの実行結果}より,表\ref{pythonでの四則演算子}
  で示した通りに動作していることがわかる.よって指導書に示されたサンプルプログラムは
  正しく動作したと言える.

  \subsection{九九表の出力(6.10.2)}
   本実験課題では図\ref{出力したい九九の表}のような九九表を出力するプログラムを作成する.

  \begin{figure}[H]
    \begin{center}
    \begin{screen}
    \begin{verbatim}
       |  1  2  3  4  5  6  7  8  9
    ---+---------------------------
      1|  1  2  3  4  5  6  7  8  9
      2|  2  4  6  8 10 12 14 16 18
      3|  3  6  9 12 15 18 21 24 27
      4|  4  8 12 16 20 24 28 32 36
      5|  5 10 15 20 25 30 35 40 45
      6|  6 12 18 24 30 36 42 48 54
      7|  7 14 21 28 35 42 49 56 63
      8|  8 16 24 32 40 48 56 64 72
      9|  9 18 27 36 45 54 63 72 81
    \end{verbatim}
    \end{screen}
    \end{center} 
    \caption{出力したい九九の表} 
    \label{出力したい九九の表}
  \end{figure}

  \subsubsection{アイデア}
  図\ref{出力したい九九の表}の最初の2行はそれぞれ掛ける数と区切りの横線を出力
  しているだけである.3行目以降は掛ける数と同じような形で九九表が出力されているため,
  指導書の指定の通りにフォーマットを使うとコンパクトなコーディングができる.
  また,各数値は1桁のものでも2桁のものでも2桁相当の空間を取って出力するため,これも
  フォーマットを活用できる.

  \subsubsection{プログラム}
  アイデアに則って組んだ九九表を出力するプログラムを以下に示す.

  \begin{figure}[H]
  \begin{lstlisting}[title={九九表を出力するプログラム}]
    # フォーマット
    f = " {0} |"
    n = " {0:2}"

    # 1行目
    print(f.format(" "),end='')
    for i in range(1,10):
        st = " " + (str)(i)
        print(n.format(st),end='')

    # 2行目
    print("\n"+"---+----------------------------")

    # 3行目以降
    for i in range(1,10):
        print(f.format(i),end='')
        for j in range(1,10):
            print(n.format(i*j),end='')
        print()
  \end{lstlisting}
  \end{figure}

  コメントにある通り,2~3行目がフォーマット,6~9行目が出力の1行目,
  12行目が出力の2行目,15~19行目が出力の3行目以降を処理している.

  \subsubsection{結果}
  図\ref{k001.pyの実行結果}に実行結果を示す.図\ref{k001.pyの実行結果}の九九表
  が図\ref{出力したい九九の表}と一致していることから,プログラムは正しく動作したと言える.
  
  \begin{figure}[H]
    \begin{center}
    \begin{screen}
    \begin{verbatim}
    >python k001.py
       |  1  2  3  4  5  6  7  8  9
    ---+---------------------------
      1|  1  2  3  4  5  6  7  8  9
      2|  2  4  6  8 10 12 14 16 18
      3|  3  6  9 12 15 18 21 24 27
      4|  4  8 12 16 20 24 28 32 36
      5|  5 10 15 20 25 30 35 40 45
      6|  6 12 18 24 30 36 42 48 54
      7|  7 14 21 28 35 42 49 56 63
      8|  8 16 24 32 40 48 56 64 72
      9|  9 18 27 36 45 54 63 72 81
    \end{verbatim}
    \end{screen}
    \end{center} 
    \caption{k001.pyの実行結果}
    \label{k001.pyの実行結果}
  \end{figure}

  \subsection{閏年判定(6.10.3)}
   本実験課題では入力された年が閏年かどうかを判定するプログラムを作成する.

  \subsubsection{アイデア[1]}
  Pythonにはcalenderモジュールというものがあり,これを用いることで
  1行のコードで与えられた数値型の年が閏年かどうかを判定できる.このモジュールを
  用いて本課題を解決する.

  \subsubsection{コード}
  前述のcalenderモジュールを用いて作成したコードを以下に示す.

  \begin{figure}[H]
  \begin{lstlisting}[title={閏年判定プログラム}]
    import calendar

    year = input("何年(YYYY):")
    year_i = int(year)

    if calendar.isleap(year_i):
        print(year_i , "年は閏年です。")
    else:
        print(year_i , "年は平年です。")
  \end{lstlisting}
  \end{figure}

  コードの1行目ではcalenderモジュールをインポートしており,3行目で年数の入力処理,
  4行目で入力の数値型変換,6行目以降でcalenderモジュールを用いた閏年判定と出力
  を行っている.

  \subsubsection{結果}
  以下の図\ref{k011.pyの実行結果}にプログラムで示したコードを実行した結果を示す.
  平年と閏年をそれぞれ入力し,正しい結果が出力されるか試した.

  \begin{figure}[H]
    \begin{center}
    \begin{screen}
    \begin{verbatim}
    > python test.py
    何年(YYYY):2025
    2025 年は平年です。
    > python test.py
    何年(YYYY):2020
    2020 年は閏年です。
    \end{verbatim}
    \end{screen}
    \end{center} 
    \caption{k011.pyの実行結果}
    \label{k011.pyの実行結果}
  \end{figure}

  図\ref{k011.pyの実行結果}の出力結果から,閏年を閏年,平年を平年と出力できたことが
  わかる.このことから本実験のプログラムは正しく動作したと言える.

  \subsection{ファイル操作(6.10.4)}
   本実験ではtemperature.datファイルを読み込み,昇順か降順か指定された順にソート
  するプログラムを作成する.

  \subsubsection{アイデア}
  本実験課題では入力に読み込むファイル(temperature.dat)と昇順か降順かを表す
  値(0,1)を入れる.次にそのdatファイルを読み込み,":"で区切られている日付と数値
  変換した温度とをセット(配列)として配列に格納する.その配列の中身を指定された
  通りにソートし,ファイルの更新を行う.また,指導書のコマンドプロンプト出力例に
  変更されたファイルの中身があったため,念のためprintによる出力も行う.本実験での
  ファイル操作は指導書の指定通りにwith文を用いて行うこととする.

  \subsubsection{プログラム}
  以下にアイデアに準じたプログラムを示す.

  他実験と同様に,本ソートプログラムの説明を各行で行う.1行目のdata配列は前述した
  セット(配列)を格納する配列である.3,4行目はファイルの名前とソート順の入力である.
  6~11行目は入力されたファイルをwith文で閲覧('r')し,for分を用いて行ごとに処理を
  行っている.処理の中身としては":"の前後で情報を区切り,それらをdate,tempとして
  data配列の中に格納している.ここで,tempはソートの際に数値として扱うためfloat型に
  変換している.15行目はコメントにある通り,ソート順を表す入力のRuleが1ならTrue,
  その他ならfalseとする変数を定義している.18行目はそれらを用いてソートを行っている.
  keyには扱うデータを参照させるのだが,データの位置を関数の出力しか指定できない.
  そのため簡易関数lambdaを使用してdataの1個目,つまり温度(0個目が日付)を参照して
  いる.21,22行目はコマンドプロンプト用にソートされたdata配列の中身を":"を挟み
  printで出力している.25~27行目はwith文の書き込み('W'で)datファイルを開き
  data配列の中身を上書きで書き込んでいる.

  \begin{figure}[H]
  \begin{lstlisting}[title={ソートプログラム}]
    data = []

    file_path = input("ファイル:")
    Rule = input("ルール(昇順:0,降順:1)")

    with open(file_path, 'r', encoding='utf-8') as f:
        for line in f:
            parts = line.split(':')
            date = parts[0]
            temp = float(parts[1])
            data.append((date, temp))

    #ソート処理 (昇順: 0, 降順: 1)
    # reverse=True で降順、False で昇順
    is_reverse = True if Rule == "1" else False

    #温度を基準にソート
    data.sort(key=lambda x: x[1], reverse=is_reverse)

    #出力処理
    for date, temp in data:
        print(f"{date}: {temp}")

    #ファイル書き換え(上書き)
    with open(file_path, 'w', encoding='utf-8') as f:
        for date, temp in data:
            f.write(f"{date}: {temp}\n")
  \end{lstlisting}
  \end{figure}

  \subsubsection{結果}
  まず指導書の出力例に則ってコマンドプロンプト上の出力結果を示す.
  最初にtempareture.datと1を入力し降順ソートを行った結果を図
  \ref{k023.pyの実行結果_cmd_1}に示す.

  \begin{figure}[H]
    \small
    \begin{center}
    \begin{screen}
    \begin{verbatim}
    > python test.py
    ファイル:temperature.dat
    ルール(昇順:0,降順:1)1
    2015/Aug/08: 99.27
    2015/Aug/15: 39.65
    2015/Aug/02: 39.56
    2015/Aug/23: 39.25
    2015/Aug/20: 39.06
    2015/Aug/06: 37.93
    2015/Aug/26: 37.47
    2015/Aug/24: 37.12
    2015/Aug/19: 36.77
    2015/Aug/27: 36.48
    2015/Aug/05: 36.23
    2015/Aug/10: 35.71
    2015/Aug/29: 35.48
    2015/Aug/30: 35.39
    2015/Aug/17: 35.22
    2015/Aug/12: 34.75
    2015/Aug/09: 34.63
    2015/Aug/11: 34.52
    2015/Aug/25: 33.77
    2015/Aug/16: 33.62
    2015/Aug/07: 33.24
    2015/Aug/13: 32.59
    2015/Aug/03: 32.38
    2015/Aug/04: 32.36
    2015/Aug/18: 31.81
    2015/Aug/21: 31.65
    2015/Aug/01: 31.34
    2015/Aug/28: 31.13
    2015/Aug/14: 31.07
    2015/Aug/31: 30.12
    2015/Aug/22: 12.58
    \end{verbatim}
    \end{screen}
    \end{center} 
    \caption{k023.pyの実行結果(cmd)}
    \label{k023.pyの実行結果_cmd_1}
  \end{figure}

  図\ref{k023.pyの実行結果_cmd_1}を見るとコマンドプロンプトへの出力は
  正しく降順ソートされたと言える.次にこの状態のままもう一度k023を実行し
  tempareture.datと0を入力し昇順ソートを行ったコマンドプロンプトの結果を
  図\ref{k023.pyの実行結果_cmd_0}に示す.

  \begin{figure}[H]
    \small
    \begin{center}
    \begin{screen}
    \begin{verbatim}
    > python test.py
    ファイル:temperature.dat
    ルール(昇順:0,降順:1)0
    2015/Aug/22: 12.58
    2015/Aug/31: 30.12
    2015/Aug/14: 31.07
    2015/Aug/28: 31.13
    2015/Aug/01: 31.34
    2015/Aug/21: 31.65
    2015/Aug/18: 31.81
    2015/Aug/04: 32.36
    2015/Aug/03: 32.38
    2015/Aug/13: 32.59
    2015/Aug/07: 33.24
    2015/Aug/16: 33.62
    2015/Aug/25: 33.77
    2015/Aug/11: 34.52
    2015/Aug/09: 34.63
    2015/Aug/12: 34.75
    2015/Aug/17: 35.22
    2015/Aug/30: 35.39
    2015/Aug/29: 35.48
    2015/Aug/10: 35.71
    2015/Aug/05: 36.23
    2015/Aug/27: 36.48
    2015/Aug/19: 36.77
    2015/Aug/24: 37.12
    2015/Aug/26: 37.47
    2015/Aug/06: 37.93
    2015/Aug/20: 39.06
    2015/Aug/23: 39.25
    2015/Aug/02: 39.56
    2015/Aug/15: 39.65
    2015/Aug/08: 99.27
    \end{verbatim}
    \end{screen}
    \end{center} 
    \caption{k023.pyの実行結果(cmd)}
    \label{k023.pyの実行結果_cmd_0}
  \end{figure}

  図\ref{k023.pyの実行結果_cmd_1}からdata配列の更新とコマンドプロンプトへの
  出力が正しく行えたと言える.また,ファイルの更新が行えているかを確認する
  ために,図\ref{k023.pyの実行結果_cmd_1}時点のtempareture.datファイルと
  図\ref{k023.pyの実行結果_cmd_0}時点のtemperature.datファイルを記録していた.
  これらを図\ref{k023.pyの実行結果_file_1}と図\ref{k023.pyの実行結果_file_0}に
  示す.

  \begin{figure}[H]
    \small
    \begin{center}
    \begin{screen}
    \begin{verbatim}
    2015/Aug/08: 99.27
    2015/Aug/15: 39.65
    2015/Aug/02: 39.56
    2015/Aug/23: 39.25
    2015/Aug/20: 39.06
    2015/Aug/06: 37.93
    2015/Aug/26: 37.47
    2015/Aug/24: 37.12
    2015/Aug/19: 36.77
    2015/Aug/27: 36.48
    2015/Aug/05: 36.23
    2015/Aug/10: 35.71
    2015/Aug/29: 35.48
    2015/Aug/30: 35.39
    2015/Aug/17: 35.22
    2015/Aug/12: 34.75
    2015/Aug/09: 34.63
    2015/Aug/11: 34.52
    2015/Aug/25: 33.77
    2015/Aug/16: 33.62
    2015/Aug/07: 33.24
    2015/Aug/13: 32.59
    2015/Aug/03: 32.38
    2015/Aug/04: 32.36
    2015/Aug/18: 31.81
    2015/Aug/21: 31.65
    2015/Aug/01: 31.34
    2015/Aug/28: 31.13
    2015/Aug/14: 31.07
    2015/Aug/31: 30.12
    2015/Aug/22: 12.58
    \end{verbatim}
    \end{screen}
    \end{center} 
    \caption{k023.pyの実行結果(ファイル)}
    \label{k023.pyの実行結果_file_1}
  \end{figure}

  \begin{figure}[H]
    \small
    \begin{center}
    \begin{screen}
    \begin{verbatim}
    > python test.py
    ファイル:temperature.dat
    ルール(昇順:0,降順:1)0
    2015/Aug/22: 12.58
    2015/Aug/31: 30.12
    2015/Aug/14: 31.07
    2015/Aug/28: 31.13
    2015/Aug/01: 31.34
    2015/Aug/21: 31.65
    2015/Aug/18: 31.81
    2015/Aug/04: 32.36
    2015/Aug/03: 32.38
    2015/Aug/13: 32.59
    2015/Aug/07: 33.24
    2015/Aug/16: 33.62
    2015/Aug/25: 33.77
    2015/Aug/11: 34.52
    2015/Aug/09: 34.63
    2015/Aug/12: 34.75
    2015/Aug/17: 35.22
    2015/Aug/30: 35.39
    2015/Aug/29: 35.48
    2015/Aug/10: 35.71
    2015/Aug/05: 36.23
    2015/Aug/27: 36.48
    2015/Aug/19: 36.77
    2015/Aug/24: 37.12
    2015/Aug/26: 37.47
    2015/Aug/06: 37.93
    2015/Aug/20: 39.06
    2015/Aug/23: 39.25
    2015/Aug/02: 39.56
    2015/Aug/15: 39.65
    2015/Aug/08: 99.27
    \end{verbatim}
    \end{screen}
    \end{center} 
    \caption{k023.pyの実行結果(ファイル)}
    \label{k023.pyの実行結果_file_0}
  \end{figure}

  図\ref{k023.pyの実行結果_file_1},図\ref{k023.pyの実行結果_file_0}から
  ファイルの更新が適切に行えていることが分かった.

  以上のことから本プログラムは題意に沿って動作できたと言える.

  \subsection{コマンドライン引数(6.10.5)}
   本実験では複数のファイルを指定し,それぞれのファイルの文字数,単語数,行数を
  カウントしそれぞれの結果を出力する.ただしプログラム引数でファイルを指定する必要があり,
  (python k032.py fruit.dat vegetable.dat wc.txt)といったような形にする.
  この例では最後の1つを除く全てのファイルであるfruit.datとvegetable.datの2つが
  カウントされるファイルであり,最後のwc.txtがカウントした結果が格納される
  ファイルである.

  %更新行

  \newpage
  \begin{huge}
    参考文献\\\\
  \end{huge}
  % \noindent[]ページ名,\url{},2026年01/11参照.\\
  \noindent[1]Pythonでうるう年を判定・カウント・列挙,\url{https://note.nkmk.me/python-calendar-leap-year/},2026年01/11参照.\\
  
  \newpage
  \begin{huge}
    感想\\\\
  \end{huge}
  Pythonはインデントによる強制的な可読性の向上と記述スタイルの統一感により,他者の
  コードを解読する際のストレスが少ないと感じた.豊富なモジュール群によって最小限の
  記述で実装できる点は,開発効率の面で非常に強力である.特に,実行結果を即座に確認
  できるインタプリタとしての特性はデータの整理やスクレイピングといった日常的な問題
  にも適しており,将来的にデータ分析や機械学習を扱う際にも,その優位性は極めて高いと考えた.
\end{document}
